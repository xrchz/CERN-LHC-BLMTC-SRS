\documentclass{article}
\usepackage{holtexbasic}
\usepackage{alltt}
\usepackage[a4paper,margin=2cm]{geometry}
\usepackage{parskip}
\begin{document}
\section*{Introduction}
First, a warning about one quirky convention.
When I write \HOLtm{SIGMA (\m. f m) (count b)} I mean for the variable $m$ to range from $0$ up to but \emph{not including} $b$.

This file contains the statements of a few key lemmas, as well as proof sketches for the important lemmas.
How to read the proofs?
Imagine you are trying to prove the statement yourself.
There are snapshots of the ``proof state'' interspersed with commentary on what to do next to get to the next proof state.
A proof state consists of what is currently remaining to be proved, then a line of dashes (\texttt{----------}) then a numbered list of assumptions that are currently being assumed.
(Note: sometimes we might be proving a subsidiary goal, and the proof state of the whole goal isn't printed until that subgoal is finished; I try to make this clear with the commentary.)

\section*{The Model}
\begin{description}
\item[Global Parameters:]\hfill
\begin{itemize}
\item
     the width of every slice, encoded as the index, \HOLtm{p.w}, of the last shift register;\\
     the total number of shift registers is \HOLtm{SUC p.w = p.w + 1};
\item
     the input series, \HOLtm{p.input}.
\end{itemize}
\item[Each slice has:]\hfill
\begin{itemize}
\item
     \HOLtm{SUC p.w} shift registers, named \HOLtm{SR p n 0}, \HOLtm{SR p n 1}, ..., \HOLtm{SR p n p.w};
\item
     an output register \HOLtm{output n}.
\end{itemize}
\item[Each SR has a source]\hfill
\begin{itemize}
\item
     The source for \HOLtm{SR p 0 0} of slice \HOLtm{0} is the global input.
\item
     The source for \HOLtm{SR p (SUC n) 0} of slice \HOLtm{SUC n} (the output of slice \HOLtm{n}).
\item
     The source for \HOLtm{SR p (SUC n) (SUC m)} is \HOLtm{SR p (SUC n ) m} (of the same slice).
\end{itemize}
\item[Each slice updates periodically]\hfill
\begin{itemize}
\item
     Slice \HOLtm{n} updates at time \HOLtm{t} if and only if \HOLtm{∃x. t = n + ((SUC x) * ((SUC p.w) ** n))}
\item
   At time \HOLtm{0}, all shift registers have value \HOLtm{0}.
\item
   At each update time for its slice, a SR gets the value of its source at the previous time.
\item
   The output of a slice at some time is the sum of its shift registers at the same time.
\end{itemize}
\end{description}

This is all summarized by the folowing theorems.
\begin{alltt}
\HOLthm[alltt]{simplified.update_time_def}
\HOLthm[alltt]{simplified.Slice_def}
\end{alltt}
\section*{Some Small Lemmas}
\input{SRLastUpdateProof.tex}
\input{SRFirstProof.tex}
\input{SR0UntilProof.tex}
\section*{Major Lemma 1}
\input{OutputFirstProof.tex}
\section*{Major Lemma 2}
\input{OutputSourceProof.tex}
\section*{Main Theorem}
\input{SumInChunksProof.tex}
\input{OutputInputProof.tex}
\end{document}
